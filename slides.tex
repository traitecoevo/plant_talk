\documentclass[mathserif,11pt]{beamer}

% Turn off the ugly navigation symbols.
\setbeamertemplate{navigation symbols}{}

% Margins and whatnot.
\setbeamersize{text margin left=1em,text margin right=1em}

% Hoefler Text as the main text font; need to make sure it's happy
% using serif though...
\usepackage{fontspec}
\usefonttheme{serif}
\defaultfontfeatures{Mapping=tex-text} % enable -- / --- / `` / ''
\setmainfont{Hoefler Text}
\setsansfont[BoldFont=Oswald Bold,ItalicFont=Oswald Light]{Oswald}
\newcommand{\amp}{{\fontspec[Alternate=1]{Hoefler Text}\&}}
\newcommand{\QQ}{{\fontspec[Alternate=2]{Hoefler Text}Q}}

\usepackage{tabularx}
\usepackage{array}

\definecolor{verydarkgrey}{HTML}{001112}
\definecolor{fdarkblue}{HTML}{232747}
\definecolor{lightgrey}{HTML}{eeeeee}
\definecolor{grey}{rgb}{0.5, 0.5, 0.5}
\definecolor{darkgreen}{rgb}{0, 0.4, 0} % same as "darkgreen" in R

\setbeamerfont{frametitle}{size=\large}
\setbeamercolor{frametitle}{fg=black}
\setbeamercolor{titleline}{bg=black}
\setbeamercolor{title}{fg=fdarkblue}
\setbeamercolor*{date in head/foot}{fg=verydarkgrey}
\setbeamercolor{item}{fg=fdarkblue}
\setbeamerfont{item}{series=\bfseries}
% Never used, but this is nice.
\setbeamertemplate{itemize subitem}{\normalsize --}

% \useoutertheme{new}

\makeatletter  %Sets category code: http://tex.stackexchange.com/questions/8351/what-do-makeatletter-and-makeatother-do
\usepackage{dashrule}
\defbeamertemplate*{frametitle}{mine}[1][left]
{
  % Increase width of title box
  \@tempdima=\textwidth%
  \advance\@tempdima by\beamer@leftmargin%
  \advance\@tempdima by\beamer@rightmargin%
  %
  \begin{beamercolorbox}[sep=0.3cm,#1,wd=\the\@tempdima]{frametitle}
    \usebeamerfont{frametitle}%
    \vbox{}\vskip-1ex\vspace{.5ex}
    \strut\hspace{1ex}\insertframetitle\strut\par%
    \vskip-1.5ex%
    \hspace{\fill}\hdashrule{\textwidth}{.075ex}{.075ex .2ex}\hspace{\fill}
    \par\nointerlineskip \vspace{\baselineskip}
  \end{beamercolorbox}%
  \nointerlineskip%
  \vspace{-1ex}
}

\setbeamercolor*{date in head/foot}{bg=black,fg=white}




\usepackage{tikz}
\usetikzlibrary{positioning}
\usetikzlibrary{calc}
\usetikzlibrary{arrows,positioning, decorations, decorations.text}

% Define new environment for overlaying transparent text box, eg.g. for title
\usepackage[framemethod=tikz]{mdframed}
\newmdenv[tikzsetting={draw=black,fill=white,fill opacity=0.7, line width=4pt, rounded corners, inner sep=10pt, inner ysep=10pt},backgroundcolor=none,leftmargin=0,rightmargin=0,innertopmargin=4pt,skipbelow=\baselineskip,%
skipabove=\baselineskip]{TitleBox}

% macro for small text at lower right of screen, e.g. links
\usepackage[overlay,absolute]{textpos}
\newcommand\FrameText[1]{
  \begin{textblock*}{\paperwidth}(0pt,.98\textheight)
    \raggedleft \small  #1\hspace{.5em}
  \end{textblock*}}

% some formatting options for green blue slide
\tikzstyle{greenblue_bodytext} = [text width=.45\paperwidth,text badly ragged,  rounded corners, inner sep=10pt, inner ysep=10pt, fill=white, line width=2pt]

% Defines a command to make two new slides with different backgorunds,
% place two text boxes on page n top left and bottom right,

\newcommand{\greenblueslide}[6]{%
\begin{frame}[empty]
   \begin{tikzpicture}[remember picture,overlay]
     \node at (current page.center) {
       \includegraphics<1>[width=\paperwidth]{#1}
       \includegraphics<2>[width=\paperwidth]{#4}
     };
     \node[draw,anchor=north west, greenblue_bodytext, draw= green] at
       ($(current page.north west) + (.05\paperwidth, -.1\paperheight)$)
       {{\bf Plants: }\\ {\small #2}};

     \only<2>{
     \node[draw,anchor=south east, greenblue_bodytext, draw=blue] at
       ($(current page.south east) - (.05\paperwidth, -.1\paperheight)$)
       {{\bf Marine: }\\ {\small #5}};
       }
   \end{tikzpicture}
   \only<1>{ \FrameText{ {\color{grey} #3}}}
   \only<2>{ \FrameText{ {\color{grey} #6}}}
\end{frame}
}

\newcommand{\traitsummary}[9]{

\tikzstyle{boxStyle1}=[anchor = center, rectangle, rounded corners, thick, inner sep=4pt, inner ysep=4pt, align = center, fill =white, text width = 3cm]
\tikzstyle{boxStyle2}=[boxStyle1, text width = 8cm]

% define styles - lines
\tikzstyle{lineStyle1}=[shorten <=2pt, shorten >=2pt]

  % anchor
  \node[] at ($(current page.center) + (0, 3cm)$) (middle){};
  % top row
   \node[boxStyle1, left = 3cm of middle.center, text = darkgreen] (low) { #1  };
   \node[boxStyle1, right = 3cm of middle.center, text = darkgreen] (high) { #3 }
      edge [<->, line width = 5pt, lineStyle1, draw=darkgreen!50]                  (low);
  \node[boxStyle1 , draw = black, text width = 3cm] at (middle) {\large #2};
  % second row
  \node[boxStyle2, below = 0.8cm of middle.center, text = black!50] (middle2) {\small (direct physiological trade-off)};
   \node[boxStyle1, left = 3cm of middle2.center] (low2) {\small #4};
   \node[boxStyle1, right = 3cm of middle2.center] (high2) {\small #5};
  % 3rd row
  \node[boxStyle2, below =0.8cm of middle2.center, text = black!50] (middle3) {\small (functional outcome)};
   \node[boxStyle1, left = 3cm of middle3.center, text = black] (low3) {\small #6};
   \node[boxStyle1, right = 3cm of middle3.center, text = black] (high3) {\small #7};
 % 4th row
  \node[boxStyle2, below =0.8cm of middle3.center, text = black!50] (middle4) {\small (demographic outcome)};
   \node[boxStyle1, left = 3cm of middle4.center, text = black] (low4) {\small #8};
   \node[boxStyle1, right = 3cm of middle4.center, text = black] (high4) {\small #9};

}


% From Rich - macro for???
\makeatother   % Sets category code: http://tex.stackexchange.com/questions/8351/what-do-makeatletter-and-makeatother-do
\usepackage{relsize}
\newenvironment{tframe}{
  \begin{frame}[plain]
    \begin{tikzpicture}[remember picture,overlay]
      \node[at=(current page.center)] {
        \includegraphics[width=\paperwidth]{pics/purple-gradient-background}
      };
    \end{tikzpicture}
    \color{white}
    \sf\relsize{3}}
    {\end{frame}}


\usepackage{pdfpages}
\usepackage{tex/fontawesome} % nice glyphs

\title{Introduction to the plant model}
\date{}
\author{}

%----------------------------------------------------------------------------------------
\begin{document}
%-----------------------------------------------------------------------------------

 \begin{frame}[plain]
 \vspace{10em}
 \begin{TitleBox}
  {\LARGE \inserttitle} \vskip3pt
  {\footnotesize \insertdate\vskip6pt
  \insertauthor }
 \end{TitleBox}
 \end{frame}


% NOTES:
% 1. Consider species as vectors of traits
% 2. Fitness is the expected number of offspring per individual, but
%    varies depending on the number and types of other individuals.
% 3. Throw in a bunch of seeds, and see how many seeds each seed
%    produces.  Consider traits to be fixed through this part.  This
%    is our "ecological black box"
\begin{frame}{Evolutionary assembly}
  \begin{center}
    \includegraphics<1>[height=.8\textheight]{figures/fitness-1}
    \includegraphics<2>[height=.8\textheight]{figures/fitness-2}
    \includegraphics<3>[height=.8\textheight]{figures/fitness-3}
  \end{center}
\end{frame}

% NOTES:
% 1. A seed is the natural place to census how types are doing.
% 2. Seeds are discrete things, and we can count them up.
% 3. A plant is simply a seeds way of making more seeds, and we can
%    largely ignore them when considering how species are doing (how
%    important is a large plant that will never reproduce).
% 4. However, they are fairly important ecologically...
% 5. To produce a new seed, the seed must
%    - germinate
%    - grow to maturity (and bigger plants make more seed)
%    - have enough spare energy to go into reproduction
% 6. So most of the ecology will focus on the plant.
% 7. But this cycle really is all that is in the black box.

\begin{frame}{Where does fitness come from?}
  \begin{center}
    \includegraphics<1>[height=.8\textheight]{figures/lifecycle-1}
    \includegraphics<2>[height=.8\textheight]{figures/lifecycle-2}
  \end{center}
\end{frame}

% NOTES:
% 1. *A plant* has a size (height, allometric scaling for
%    other sizes), probability of dying and a rate of seed production.
% 2. All these rates are functions of *total carbon assimilation*,
%    which is a function of the plant's leaf area distribution (canopy
%    shape) and the light environment.
% 3. The carbon fixed is then assigned to maintenance, growth and
%    reproduction.  Your probability of mortality depends inversely on
%    the amount of photosynthesis you do.
% 4. (The key ingredient is how much carbon you fix; we consider light
%    primarily as the limiting resource)
\begin{frame}{Inside our black box}
  \begin{center}
    \includegraphics<1>[height=.8\textheight]{figures/plantmodel-1}
    \includegraphics<2>[height=.8\textheight]{figures/plantmodel-2}
    \includegraphics<3>[height=.8\textheight]{figures/plantmodel-3}
    \includegraphics<4>[height=.8\textheight]{figures/plantmodel-4}
    \includegraphics<5>[height=.8\textheight]{figures/plantmodel-5}
    \includegraphics<6>[height=.8\textheight]{figures/plantmodel-6}
    \includegraphics<7>[height=.8\textheight]{figures/plantmodel-7}
  \end{center}
\end{frame}

% NOTES:
% The light environment is derived by summing over all the leaf
% area distributions of all individuals -- it also varies over time
% and emerges from the model.  It provides a strong feedback between
% all individuals in the model.
%
% [Plants grow in the context of a forest; how much light you get
% depends on who grows around you and how big they are; but that
% depends on who grows around those plants, and so on.  This is the
% key feedback]
\begin{frame}{Strong feedbacks via the light environment}
  \begin{center}
    \includegraphics<1>[height=.8\textheight]{figures/lightenv-1}
    \includegraphics<2>[height=.8\textheight]{figures/lightenv-2}
    \includegraphics<3>[height=.8\textheight]{figures/lightenv-3}
  \end{center}
\end{frame}

% \begin{frame}{Growth dynamics}
%   \begin{center}
%     \includegraphics<1>[height=.8\textheight]{figures/growth-1}
%     \includegraphics<2>[height=.8\textheight]{figures/growth-2}
%     \includegraphics<3>[height=.8\textheight]{figures/growth-3}
%   \end{center}
% \end{frame}

\begin{frame}{Strong ecological feedbacks dominate this model}
  \vspace{-1em}
  \begin{center}
    \includegraphics[height=.7\textheight]{figures/blackbox}
  \end{center}
\end{frame}

\begin{frame}{Evolutionary assembly}
  \begin{center}
    \includegraphics<1>[height=.8\textheight]{figures/fitness-4}
    \includegraphics<2>[height=.8\textheight]{figures/fitness-3}
  \end{center}
\end{frame}

\end{document}
